\documentclass{article}

\usepackage[utf8]{inputenc}
\usepackage[OT1]{fontenc}
\usepackage[french]{babel}
\usepackage{tabto}

\begin{document}

LS(1)                            User Commands                           LS(1) \\
\\
NAME \\
       ls - list directory contents \\
\\
SYNOPSIS \\
       ls [OPTION]... [FILE]... \\
\\
DESCRIPTION \\
\tabto{1cm}      List  information  about  the FILEs (the current directory by default).
\tabto{1cm}       Sort entries alphabetically if none of -cftuvSUX nor --sort  is  speci‐
\tabto{1cm}       fied. \\
\\
 \tabto{1cm}      Mandatory  arguments  to  long  options are mandatory for short options
       too. \\
\\
\tabto{1cm}       -a, \--all\\
\tabto{2cm}              do not ignore entries starting with .\\
\\
\tabto{1cm}       -A, \--almost-all\\
\tabto{2cm}              do not list implied . and .. \\
\\
 \tabto{1cm}      - -author \\
 \tabto{2cm}             with -l, print the author of each file \\
\\
\tabto{1cm}      -b, \--escape \\
\tabto{2cm}              print C-style escapes for nongraphic characters \\
\\
\tabto{1cm}       - -block-size=SIZE \\
\tabto{2cm}              with  -l,  scale  sizes  by  SIZE  when  printing  them;   e.g.,
\tabto{2cm}              '- -block-size=M'; see SIZE format below \\
\\
\tabto{1cm}       -B, \--ignore-backups \\
\tabto{2cm}              do not list implied entries ending with ~ \\
\\
\tabto{1cm}       -c     with -lt: sort by, and show, ctime (time of last modification of
\tabto{2cm}              file status information); with -l: show ctime and sort by  name;
\tabto{2cm}              otherwise: sort by ctime, newest first \\
\\
\tabto{1cm}       -C     list entries by columns \\
\\
\tabto{1cm}       - -color[=WHEN] \\
\tabto{2cm}              colorize  the output; WHEN can be 'always' (default if omitted),
\tabto{2cm}              'auto', or 'never'; more info below \\
\\
\\
\tabto{1cm}       -d, \--directory \\
\tabto{2cm}              list directories themselves, not their contents \\
\\
\tabto{1cm}       -D, \--dired \\
\tabto{2cm}              generate output designed for Emacs' dired mode \\
\\
\tabto{1cm}       -f     do not sort, enable -aU, disable -ls --color \\
\\
\tabto{1cm}       -F, \--classify \\
\tabto{2cm}              append indicator (one of */=>@|) to entries \\
\\
\tabto{1cm}       - -file-type \\
\tabto{2cm}              likewise, except do not append '*' \\
\\
\tabto{1cm}       - -format=WORD \\
\tabto{2cm}              across -x, commas -m, horizontal -x, long -l, single-column  -1,
              verbose -l, vertical -C \\
\\
\tabto{1cm}       - -full-time \\
\tabto{2cm}              like -l - -time-style=full-iso \\
\\
\tabto{1cm}       -g     like -l, but do not list owner \\
\\
\tabto{1cm}       --group-directories-first \\
\tabto{2cm}              group directories before files; \\
\\
\tabto{2cm}              can   be  augmented  with  a  --sort  option,  but  any  use  of
\tabto{2cm}              --sort=none (-U) disables grouping \\
\\
\tabto{1cm}       -G, --no-group \\
\tabto{2cm}              in a long listing, don't print group names \\
\\
\tabto{1cm}      -h, --human-readable \\
\tabto{2cm}              with -l and -s, print sizes like 1K 234M 2G etc. \\
\\
\tabto{1cm}       --si   likewise, but use powers of 1000 not 1024 \\
\\
\tabto{1cm}       -H, --dereference-command-line \\
\tabto{2cm}              follow symbolic links listed on the command line \\
\\
\tabto{1cm}       --dereference-command-line-symlink-to-dir \\
\tabto{2cm}              follow each command line symbolic link \\
\\
\tabto{2cm}              that points to a directory \\
\\
\tabto{1cm}       --hide=PATTERN \\
\tabto{2cm}              do not list implied entries matching shell  PATTERN  (overridden
\tabto{2cm}              by -a or -A) \\
\\
\tabto{1cm}       --hyperlink[=WHEN] \\
\tabto{2cm}              hyperlink file names; WHEN can be 'always' (default if omitted),
\tabto{2cm}              'auto', or 'never' \\
\\
\tabto{1cm}       --indicator-style=WORD \\
\tabto{2cm}              append indicator with style WORD to entry names: none (default),
\tabto{2cm}              slash (-p), file-type (--file-type), classify (-F) \\
\\
\tabto{1cm}       -i, --inode \\
\tabto{2cm}              print the index number of each file \\
\\
\tabto{1cm}       -I, --ignore=PATTERN \\
\tabto{2cm}              do not list implied entries matching shell PATTERN \\
\\
\tabto{1cm}       -k, --kibibytes \\
\tabto{2cm}              default  to  1024-byte  blocks for disk usage; used only with -s
\tabto{2cm}              and per directory totals \\
\\
\tabto{1cm}       -l     use a long listing format \\
\\
\tabto{1cm}       -L, --dereference \\
\tabto{2cm}              when showing file information for a symbolic link, show informa‐
\tabto{2cm}              tion  for  the file the link references rather than for the link
\tabto{1cm}              itself \\
\\
\tabto{1cm}       -m     fill width with a comma separated list of entries \\
\\
\tabto{1cm}       -n, --numeric-uid-gid \\
\tabto{2cm}              like -l, but list numeric user and group IDs \\
\\
\tabto{1cm}       -N, --literal \\
\tabto{2cm}              print entry names without quoting \\
\\
\tabto{1cm}       -o     like -l, but do not list group information \\
\\
\tabto{1cm}       -p, --indicator-style=slash \\
\tabto{2cm}              append / indicator to directories \\
\\
\tabto{1cm}       -q, --hide-control-chars \\
\tabto{2cm}             print ? instead of nongraphic characters \\
\\
\tabto{1cm}       - -show-control-chars \\
\tabto{2cm}              show nongraphic characters as-is (the default, unless program is
\tabto{2cm}              'ls' and output is a terminal) \\
\\
\tabto{1cm}       -Q, - -quote-name \\
\tabto{2cm}              enclose entry names in double quotes \\
\\
\tabto{1cm}       - -quoting-style=WORD \\
\tabto{2cm}              use  quoting style WORD for entry names: literal, locale, shell,
\tabto{2cm}              shell-always,  shell-escape,  shell-escape-always,   c,   escape
\tabto{2cm}              (overrides QUOTING_STYLE environment variable) \\
\\
\tabto{1cm}       -r, - -reverse \\
\tabto{2cm}              reverse order while sorting \\
\\
\tabto{1cm}       -R, - -recursive \\
\tabto{2cm}              list subdirectories recursively \\
\\
\tabto{1cm}       -s, --size \\
\tabto{2cm}              print the allocated size of each file, in blocks \\
\\
\tabto{1cm}       -S     sort by file size, largest first \\
\\
\tabto{1cm}       --sort=WORD \\
\tabto{2cm}              sort  by  WORD instead of name: none (-U), size (-S), time (-t),
\tabto{2cm}              version (-v), extension (-X) \\
\\
\tabto{1cm}       --time=WORD \\
\tabto{2cm}              change the default of  using  modification  times;  access  time
\tabto{2cm}              (-u): atime, access, use; change time (-c): ctime, status; birth
\tabto{2cm}              time: birth, creation; \\
\\
\tabto{2cm}             with -l, WORD determines which time to show;  with  --sort=time,
\tabto{2cm}              sort by WORD (newest first) \\
\\
\tabto{1cm}       --time-style=TIME_STYLE \\
\tabto{2cm}              time/date format with -l; see TIME_STYLE below \\
\\
\tabto{1cm}       -t     sort by time, newest first; see --time \\
\\
\tabto{1cm}       -T, --tabsize=COLS \\
\tabto{2cm}              assume tab stops at each COLS instead of 8 \\
\\
\tabto{1cm}       -u     with  -lt:  sort by, and show, access time; with -l: show access
\tabto{2cm}              time and sort by name; otherwise: sort by  access  time,  newest
\tabto{2cm}              first \\
\\
\tabto{1cm}       -U     do not sort; list entries in directory order \\
\\
\tabto{1cm}       -v     natural sort of (version) numbers within text \\
\\
\tabto{1cm}       -w, --width=COLS \\
\tabto{2cm}              set output width to COLS.  0 means no limit \\
\\
\tabto{1cm}       -x     list entries by lines instead of by columns \\
\\
\tabto{1cm}       -X     sort alphabetically by entry extension \\
\\
\tabto{1cm}       -Z, --context \\
\tabto{2cm}              print any security context of each file \\
\\
\tabto{1cm}       -1     list one file per line.  Avoid '\n' with -q or -b \\
\\
\tabto{1cm}       --help display this help and exit \\
\\
\tabto{1cm}       --version \\
\tabto{2cm}              output version information and exit \\
\\
\tabto{1cm}       The  SIZE  argument  is  an  integer and optional unit (example: 10K is
\tabto{1cm}       10*1024).  Units are K,M,G,T,P,E,Z,Y  (powers  of  1024)  or  KB,MB,...
\tabto{1cm}       (powers  of 1000).  Binary prefixes can be used, too: KiB=K, MiB=M, and
\tabto{1cm}       so on. \\
\\
\tabto{1cm}       The TIME_STYLE argument can be  full-iso,  long-iso,  iso,  locale,  or
\tabto{1cm}       +FORMAT.   FORMAT  is  interpreted  like in date(1).  If FORMAT is FOR‐
\tabto{1cm}       MAT1<newline>FORMAT2, then FORMAT1 applies to non-recent files and FOR‐
\tabto{1cm}       MAT2  to  recent files.  TIME_STYLE prefixed with 'posix-' takes effect
\tabto{1cm}       only outside the POSIX locale.  Also the TIME_STYLE  environment  vari‐
\tabto{1cm}       able sets the default style to use. \\
\\
\tabto{1cm}       Using  color  to distinguish file types is disabled both by default and
\tabto{1cm}       with --color=never.  With --color=auto, ls emits color codes only  when
\tabto{1cm}       standard  output is connected to a terminal.  The LS_COLORS environment
\tabto{1cm}       variable can change the settings.  Use the dircolors command to set it.
\\
   Exit status: \\
\tabto{1cm}       0      if OK, \\
\\
\tabto{1cm}       1      if minor problems (e.g., cannot access subdirectory), \\
\\
\tabto{1cm}       2      if serious trouble (e.g., cannot access command-line argument). \\
\\
AUTHOR \\
\tabto{1cm}       Written by Richard M. Stallman and David MacKenzie. \\
\\
REPORTING BUGS \\
\tabto{1cm}       GNU coreutils online help: <https://www.gnu.org/software/coreutils/>
\tabto{1cm}       Report any translation bugs to <https://translationproject.org/team/> \\
\\
COPYRIGHT \\
\tabto{1cm}       Copyright \copyright 2020 Free Software Foundation, Inc.   License  GPLv3+:  GNU \\
\tabto{1cm}       GPL version 3 or later <https://gnu.org/licenses/gpl.html>. \\
\tabto{1cm}       This  is  free  software:  you  are free to change and redistribute it. \\
\tabto{1cm}       There is NO WARRANTY, to the extent permitted by law. \\
\\
SEE ALSO \\
\tabto{1cm}       Full documentation <https://www.gnu.org/software/coreutils/ls> \\
\tabto{1cm}       or available locally via: info '(coreutils) ls invocation' \\
\\
GNU coreutils 8.32              September 2020                           LS(1) \\

\end{document}
